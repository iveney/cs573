% ---------
%  Compile with "pdflatex hw0".  
% --------
%!TEX TS-program = pdflatex

\documentclass[letterpaper,11pt]{article}
\usepackage{jeffe,handout,graphicx}
\usepackage{fancyhdr}
\usepackage{comment}
\graphicspath{{./Fig/}}

\bibliographystyle{unsrt}
% ---------
% Input file uses Unicode's UTF-8 encoding
% ---------
%!TEX encoding = UTF-8 Unicode
\usepackage[T1]{fontenc}
\usepackage[utf8]{inputenc}

% ---------
%  The next several lines (up to the line of =='s) change the default text
%  and math fonts and make a few other minor cosmetic changes.  If you get
%  any error messages related to these packages, just comment them out.
%         -- Jeff
% --------
\usepackage[charter]{mathdesign}
 \def\sfdefault{fvs}
 \def\ttdefault{fvm}
 \SetMathAlphabet{\mathsf}{bold}{\encodingdefault}{\sfdefault}{b}{\updefault}
 \SetMathAlphabet{\mathtt}{bold}{\encodingdefault}{\ttdefault}{b}{\updefault}
 \SetMathAlphabet{\mathsf}{normal}{\encodingdefault}{\sfdefault}{\mddefault}{\updefault}
 \SetMathAlphabet{\mathtt}{normal}{\encodingdefault}{\ttdefault}{\mddefault}{\updefault}
 \usepackage{microtype}
% ---------
%  End of cosmetics.
% --------

\newcommand{\name}{Zigang Xiao (zxiao2), Yuelin Du (du6), Pei-Ci Wu (peiciwu1)} 
\newcommand{\hwnumber}{5}                % fill homework count
\newcounter{probid}
\newtheorem{definition}{Definition}
\newtheorem{theorem}{Theorem}
\newcommand{\hdr}[2]{
   \newpage\setcounter{page}{1}       % reset page counter
   \lhead{\fancyplain{}{\textbf{#1}}}
   \rhead{\fancyplain{}{\textbf{#2}}}
   \cfoot{\fancyplain{}{\thepage}}
}

\newcommand{\stmt}{
 \EMPH{I understand the course policies.}
}

\newcommand{\saystmt}{
($\bullet$) \EMPH{I understand the course policies.}
}

\newcommand{\newprob}{
\hdr{CS 573 HW\hwnumber}{\name{} HW\hwnumber{} \#\arabic{probid}}
\stepcounter{probid}
\item
}

% =========================================================
\begin{document}
\pagestyle{fancy}
\fancyhf{}
\small\sf	% typeset excetps from problems in a different, smaller font

\begin{enumerate}
% ---------------------------------------------------------
\newprob
\saystmt

Describe and analyze an algorithm to nest the boxes so that the number of
visible boxes is as small as possible.

\begin{solution}
  We assume one box can only contain \emph{exactly} one another box.
  We reduce this problem to bipartite matching problem, and then 
  use Ford-Fulkerson algorithm to solve it.
  We construct a bipartite graph $G$ with vertex set $U \cup W$ as follows:
  \begin{itemize}
    \item Each box is a node in both $U$ and $W$.
    \item For a pair of node $u\in U$ and $v \in W$, if $u$ can be rotated so 
      that it can be contained in $v$, we add an edge $e=(u,v)$.
  \end{itemize}

  Call two boxes are nested if one contains another.
  Let $N$ be the maximum number of nested boxes.
  Then, the number of visible boxes can be simply computed as $n-N$.

  \begin{theorem}
    The maximum number of nested boxes equals to $|M|$, where $|M|$ is the
    value of the size of the maximum matching in $G$.
  \end{theorem}

  \begin{proof}
    If we know the maximum number of nested boxes and how the boxes contain
    each other, we can transfer to a maximum matching in $G$ as follows: 
    if $u$ contains $v$, then we select $e=(u,v)$ in $G$, where $u\in U$ and
    $v\in W$.  Each time we place a box into another, the number of visible
    boxes is decreased by one.  
    Let $|M|$ be the final number of edges selected.
    It is maximized according to the above procedure.
    Otherwise, there will be a better way to place the boxes. 
    Since each box can contain exactly one another box, this is a valid
    matching.  Moreover, the number of the nested boxes and the size of the
    maximum matching is $|M|$. 

    Conversely, if we know a maximum matching, we can find how to place the
    boxes into each other and then get the smallest number of visible boxes.
    If we match $u$ with $v$, then we place box $u$ into $v$.
    Finally, the size of the maximum matching is $|M|$, and equals to the
    number of nested boxes. 
  \end{proof}

  There are at most six permutation of width, height and depth. Hence for a
  pair of boxes, we can determine whether they can contain one another in
  constant time. Let there be $n$ boxes. Then, there are $2n$ nodes and at most
  $O(n^2)$ edges in $G$.  Hence, the maximum flow has value at most $O(n)$, and
  the Ford-Fulkerson algorithm runs in $O(n^3)$ time.

\end{solution}

% ---------------------------------------------------------
\newprob
\saystmt

Describe an efficient algorithm that either rounds $A$ in this fashion, or
reports correctly that no such rounding is possible.

\begin{solution}
  test
\end{solution}
% ---------------------------------------------------------

% ---------------------------------------------------------
\newprob
\saystmt

Describe and analyze an algorithm to compute a donation schedule, describing
how much money each voter should send to each candidate on each day, that
guarantees that every candidate gets enough money to win their election.  The
schedule must obey both Federal laws and individual voters' budget constraints.
If no such schedule exists, your algorithm should report that fact.

\begin{solution}
  test
\end{solution}
% ---------------------------------------------------------
\newprob

\begin{enumerate}
  \item[($\bullet$)] \stmt
  \item 
Prove that this greedy strategy does not always compute an optimal solution.

\begin{solution}
  test
\end{solution}

\item
Describe and analyze an efficient algorithm to compute the smallest set of
monotone paths that covers every marked cell. The input to your algorithm is an
array $M[1..n,1..n]$ of booleans, where $M[i, j]$ = TRUE if and only if cell
$(i, j)$ is marked.

\begin{solution}
  test
\end{solution}

\end{enumerate}


% ---------------------------------------------------------
\newprob

\begin{enumerate}
  \item[($\bullet$)] \stmt
  \item 
    Prove that if Paul uses a deterministic strategy, and Sally knows his
    strategy, then Sally can guarantee that she wins.
  \item 
    Describe a deterministic strategy for Sally that guarantees that she wins
    when $r \leq M$, no matter what strategy Paul uses.
  \item 
    Prove that if Sally uses a deterministic strategy, and Paul knows her
    strategy then Paul can guarantee that he wins when $r < M$.
  \item 
    Describe a randomized strategy for Sally that guarantees that she wins with
    probability at least $\min{r/M,1}$, no matter what strategy Paul uses.
  \item 
    Describe a randomized strategy for Paul that guarantees that he loses with
    probability at most $\min{r/M,1}$, no matter what strategy Sally uses.

\end{enumerate}

% =========================================================

\end{enumerate}
\end{document}
